\documentclass[11pt,letterpaper,reqno]{amsart}
\usepackage{tikz}
\usetikzlibrary{positioning, shapes.geometric, arrows.meta, calc, positioning}
\usepackage{amssymb}
\usepackage{amsmath}
\usepackage{amsthm}
\usepackage{amsfonts}
\usepackage{bbm}
\usepackage{enumitem} 
\usepackage{pgfplots}
\pgfplotsset{compat=1.18} 
\usepackage{booktabs}
\usepackage{graphicx}
\usepackage[T1]{fontenc}
\usepackage{doi}
\usepackage{float} 
\addtolength{\hoffset}{-1.5cm}\addtolength{\textwidth}{3cm}
\addtolength{\voffset}{-1cm}\addtolength{\textheight}{2cm}

\usepackage{bookmark}
\usepackage{hyperref}
\hypersetup{pdfstartview={FitH}}
\newcommand{\C}{\mathbb{C}}
\newcommand{\cE}{\mathcal{E}}
\newcommand{\norm}[1]{\lVert #1 \rVert}
\newcommand{\abs}[1]{| #1 |}
\newcommand{\bv}{\mathbf{v}}
\newcommand{\bw}{\mathbf{w}}
\newcommand{\tr}{\operatorname{Tr}}
\DeclareMathOperator{\rank}{rank}




\newtheorem{theorem}{Theorem}[section]
\newtheorem{lemma}[theorem]{Lemma}
\newtheorem{proposition}[theorem]{Proposition}
\newtheorem{corollary}[theorem]{Corollary}
\newtheorem{claim}{Claim}
\newtheorem{question}[theorem]{Question}
\newtheorem{problem}[theorem]{Problem}
\newtheorem{conjecture}[theorem]{Conjecture}
\theoremstyle{definition}
\newtheorem{example}[theorem]{Example}
\newtheorem{remark}[theorem]{Remark}
\newtheorem{definition}[theorem]{Definition}
\numberwithin{equation}{section}
\newcommand{\NN}{\mathbb{N}}
\newcommand{\taufunc}{\tau}
\newcommand{\omegap}{\omega}
\newcommand{\ord}{\operatorname{ord}}
\newcommand{\R}{\mathbb{R}}        % real numbers
\newcommand{\E}{\mathbb{E}}        % expectation
\newcommand{\Var}{\mathrm{Var}}    % variance
\newcommand{\Cov}{\operatorname{Cov}}
\newcommand{\PP}{\mathbb{P}}     % probability
\newcommand{\eps}{\varepsilon}     % epsilon
\newcommand{\ind}{\mathbf{1}}      % indicator function
\newcommand{\seq}[1]{\left(#1\right)} % sequence
\newcommand{\lcm}{\operatorname{lcm}}
\makeatother


\begin{document}

% \title{}

% \author[]{}
% \address{} 
% \email{}

% \author[]{}
% \address{}
% \email{}


% \subjclass[2020]{xxx}

% \keywords{xxx}

% \begin{abstract}

% \end{abstract}



% \maketitle


\section{Introduction}

In~\cite[p.~80]{Er69}, Erd\H{o}s wrote:
\begin{quote}
\emph{I just want to state without proof a special result in this
direction, namely
\begin{equation}\label{eq:erdos_error_result_1}
\frac{\log\log n}{\log n}n+c_9\frac{n}{(\log n)^2}\leq g_3(n)\leq \frac{\log\log n}{\log n}n+c_{10}\frac{n}{(\log n)^2}.
\end{equation}
It is not clear whether~\eqref{eq:erdos_error_result_1} can be sharpened.}
\end{quote}
This leads to the following problem, which also appears as Problem~\#796 on Bloom's Erd\H{o}s Problems website~\cite{EP796}.
\begin{problem}\label{prob:EP796}
Let $k\geq 2$ and let $g_k(n)$ be the largest possible size of $A\subseteq \{1,\ldots,n\}$ such that every $m$ has $<k$ solutions to $m=a_1a_2$ with $a_1<a_2\in A$.

Is it true that\[g_3(n)=\frac{\log\log n}{\log n}n+(c+o(1))\frac{n}{(\log n)^2}\]for some constant $c$?    
\end{problem}


In this note, we show that~\eqref{eq:erdos_error_result_1} is false, thereby providing a counterexample to Problem~\ref{prob:EP796}. We also give a corrected version of the problem.










\section{A lower bound}
Fix $n\ge 2$. Then $g_3(n)$ denote the largest cardinality of a set
$A\subseteq \{1,2,\dots,n\}$ with the property that for every integer $m$
the equation
\[
m=a_1a_2,\qquad a_1<a_2,\ a_1,a_2\in A,
\]
has at most $2$ solutions.

Let $M$ be the Meissel--Mertens constant. Then we have the following lower bound.
\begin{proposition}\label{prop:main_lower_bound_1}
For all sufficiently large $n$,
\[
g_3(n) \ge \frac{n\log\log n}{\log n}+(M+1+o(1))\frac{n}{\log n}.
\]
\end{proposition}

\begin{proof}
Let
\[
P:=\{p\le n:\ p\ \text{is prime}\},
\qquad
S:=\{pq\le n:\ p,q\ \text{primes and } q>\sqrt n\},
\]
and define
\[
A:=P\cup S\subseteq \{1,2,\dots,n\}.
\]

\noindent\textbf{1. $A$ has at most two multiplicative representations. }Call a prime \emph{large} if it exceeds $\sqrt n$. 

\begin{claim}\label{claim:one-large}
Every element $a\in A$ has at most one large prime divisor (counted without multiplicity).
\end{claim}

\begin{proof}[Proof of Claim~\ref{claim:one-large}]
If $a\in P$ then $a$ is prime, so the claim is trivial.
If $a\in S$, then $a=pq$ with $q>\sqrt n$ prime and $p\le n/q<\sqrt n$, so $q$ is the unique
large prime divisor of $a$.
\end{proof}

\begin{claim}\label{claim:two-reps}
For every integer $m$, the number of solutions of $m=a_1a_2$ with $a_1<a_2$ and
$a_1,a_2\in A$ is at most $2$.
\end{claim}

\begin{proof}[Proof of Claim~\ref{claim:two-reps}]
Fix $m$ and suppose $m=a_1a_2$ with $a_1<a_2$ and $a_1,a_2\in A$.
By Claim~\ref{claim:one-large}, each $a_i$ contributes at most one large prime divisor,
so $m$ has at most two large prime divisors (counted with multiplicity).

\smallskip
\noindent\emph{Case 0: $m$ has no large prime divisor.}
Then neither $a_1$ nor $a_2$ can lie in $S$, hence $a_1,a_2\in P$ and $a_1,a_2\le \sqrt n$.
Thus $m$ is the product of two primes and the representation is unique up to order; with the
constraint $a_1<a_2$, there is at most one solution.

\smallskip
\noindent\emph{Case 1: $m$ has exactly one large prime divisor $q>\sqrt n$.}
Then exactly one of $a_1,a_2$ is divisible by $q$. The factor divisible by $q$ is either
$q\in P$ or $pq\in S$ with $p\le \sqrt n$ prime. The other factor has no large prime divisor,
hence must be a prime $\le \sqrt n$ (an element of $P$). Consequently, $m$ has the form
\[
m=qpr
\]
where $p,r\le \sqrt n$ are primes (not necessarily distinct), and any solution corresponds
to deciding whether $q$ is paired with $p$ or with $r$. This yields at most $2$ solutions
(with fewer if $p=r$).

\smallskip
\noindent\emph{Case 2: $m$ has exactly two large prime divisors (counted with multiplicity).}
Let the large prime divisors of $m$ (with multiplicity) be $q_1,q_2>\sqrt n$.
By Claim~\ref{claim:one-large}, each $a\in A$ contains at most one large prime divisor, hence in any
representation $m=a_1a_2$ with $a_1<a_2$ and $a_1,a_2\in A$, the two large primes $q_1,q_2$ must be split
between $a_1$ and $a_2$, i.e. each of $a_1,a_2$ is divisible by exactly one of $q_1,q_2$. Moreover, by the definition of $A=P\cup S$, every element of $A$ is either a prime $\le n$, or a product
$pq$ with $q>\sqrt n$ prime and $p$ prime (necessarily $p<\sqrt n$).
In particular, if $a\in A$ is divisible by a large prime $q>\sqrt n$, then
\[
a=q\cdot u\qquad\text{with }u\in\{1\}\cup\{\text{primes }\le \sqrt n\}.
\]
Therefore any solution $m=a_1a_2$ forces
\[
a_1=q_1u_1,\qquad a_2=q_2u_2
\]
(up to swapping $q_1,q_2$), where $u_1,u_2\in\{1\}\cup\{\text{primes }\le \sqrt n\}$.

If $q_1\neq q_2$, then the only possible ambiguity comes from swapping which small factor
($u_1$ or $u_2$) is paired with $q_1$ or $q_2$.
Since each $u_i$ is either $1$ or a single prime, there are at most two such pairings:
\[
(q_1u_1)(q_2u_2)\quad\text{or}\quad (q_1u_2)(q_2u_1),
\]
and after imposing $a_1<a_2$ this gives at most $2$ solutions (and fewer if $u_1=u_2$).

If $q_1=q_2=q$, then every admissible factor must contain exactly one copy of $q$, hence
$a_1=qu_1$, $a_2=qu_2$ with $u_1,u_2$ as above. Swapping yields the same unordered pair, so with the
constraint $a_1<a_2$ there is at most one solution.

Thus in all subcases, the number of solutions is at most $2$.


\smallskip
In all cases, the number of solutions is at most $2$.
\end{proof}



\noindent\textbf{2. Size of $A$. }Clearly
\[
|A|=|P|+|S|=\pi(n)+|S|.
\]Thus by Proposition~\ref{prop:lower}, we know that\[
|A|
=\pi(n)+|S|
=\frac{n\log\log n}{\log n}+(M+1)\frac{n}{\log n}+o \left(\frac{n}{\log n}\right).
\]Claim~\ref{claim:two-reps} shows that $A$ is admissible, hence \(g_3(n)\ge |A|\). This completes the proof.
\end{proof}


\section{Concluding Remarks}
In~\cite[p.~261]{Er64d}, Erd\H{o}s stated that his~\cite[Theorem~3]{Er64d} could be sharpened to
\[
g_3(n)\leq \frac{\log\log n}{\log n}n+O\!\left(\frac{n}{(\log n)^{1+c}}\right),
\]
where $c>0$ is a suitable positive constant. In view of Proposition~\ref{prop:main_lower_bound_1}, we now conjecture that the second term in the upper bound is also of order $n/\log n$:

\begin{conjecture}
There exists an absolute constant $c>0$ such that, for all sufficiently large $n$,
\[
g_3(n) \le \frac{n\log\log n}{\log n}+c\frac{n}{\log n}.
\]
\end{conjecture}

More strongly, we also conjecture the following modified version of Problem~\ref{prob:EP796}:
\begin{conjecture}
Is it true that
\[
g_3(n)=\frac{\log\log n}{\log n}n+(c+o(1))\frac{n}{\log n}
\]
for some constant $c$?
\end{conjecture}





































% \newpage



\begin{thebibliography}{99}

\bibitem{EP796}
T. F. Bloom, Erdős Problem \#796, \url{https://www.erdosproblems.com/796}, accessed 2025-12-30.

\bibitem{CrEr20}
Crisan, Dragos, and Radek Erban. "On the counting function of semiprimes." arXiv preprint arXiv:2006.16491 (2020).


\bibitem{Er64d}
P. Erd\H{o}s, On the multiplicative representation of integers. Israel Journal of Mathematics (1964), 251--261.



\bibitem{Er69}
P. Erd\H{o}s, Some applications of graph theory to number theory. The Many Facets of Graph Theory (Proc. Conf., Western Mich. Univ., Kalamazoo, Mich., 1968) (1969), 77--82.


\end{thebibliography}





\appendix


\section{Preliminaries}
Let $n\ge 3$. Define
\[
S:=\{pq\le n:\ p,q \text{ are primes and } q>\sqrt n\}.
\]
Let $\pi(x)$ be the prime-counting function, and let $\pi_2(x)$ denote the
\emph{semiprime counting function} defined by
\[
\pi_2(x):=\#\{(p,q):\ p\le q,\ p,q\ \text{prime},\ pq\le x\}.
\]


\begin{lemma}\label{lem:decomp}
For every $n\ge 3$,
\[
|S|
=\pi_2(n)-\#\{(p,q):\ p\le q\le \sqrt n,\ p,q\ \text{prime}\}
=\pi_2(n)-\frac{\pi(\sqrt n)\bigl(\pi(\sqrt n)+1\bigr)}{2}.
\]
\end{lemma}

\begin{proof}
Every pair $(p,q)$ of primes with $p\le q$ and $pq\le n$ falls into exactly one of the
two disjoint classes:
\[
\mathcal{A}:=\{(p,q):\ p\le q\le \sqrt n\},\qquad
\mathcal{B}:=\{(p,q):\ p\le q,\ pq\le n,\ q>\sqrt n\}.
\]
Hence $\pi_2(n)=|\mathcal{A}|+|\mathcal{B}|$.

If $(p,q)\in\mathcal{A}$ then automatically $pq\le (\sqrt n)^2=n$, so $\mathcal{A}$ is simply
the set of unordered pairs of primes $\le \sqrt n$ with repetition allowed. Therefore
\[
|\mathcal{A}|=\binom{\pi(\sqrt n)}{2}+\pi(\sqrt n)
=\frac{\pi(\sqrt n)\bigl(\pi(\sqrt n)+1\bigr)}{2}.
\]

If $(p,q)\in\mathcal{B}$ then $q>\sqrt n$ and $pq\le n$; this is equivalent to the condition
that the integer $pq$ lies in $S$. Moreover, because $q>\sqrt n$ implies $p<n/q<\sqrt n<q$,
each element of $S$ corresponds to a unique pair $(p,q)\in\mathcal{B}$ with $p\le q$.
Thus $|\mathcal{B}|=|S|$, and the claimed identity follows.
\end{proof}



Clearly, we have the asymptotic formula (see, for example, \cite[Theorem~2.3]{CrEr20})
\begin{equation}\label{eq:pi2-asymp}
\pi_2(x)=\frac{x\log\log x}{\log x}+M\frac{x}{\log x}+o\!\left(\frac{x}{\log x}\right).
\end{equation}
We also recall the prime number theorem $\pi(x)\sim x/\log x$. 


\begin{proposition}\label{prop:lower}
We have
\[
|S|
=\frac{n\log\log n}{\log n}+M\frac{n}{\log n}+o\!\left(\frac{n}{\log n}\right).
\]
\end{proposition}
% In particular, for every $\varepsilon>0$ there exists $n_0(\varepsilon)$ such that for all
% $n\ge n_0(\varepsilon)$,
% \[
% |S|\ge \frac{n\log\log n}{\log n}+(M-\varepsilon)\frac{n}{\log n}.
% \]


\begin{proof}
By Lemma~\ref{lem:decomp},
\[
|S|=\pi_2(n)-\frac{\pi(\sqrt n)\bigl(\pi(\sqrt n)+1\bigr)}{2}.
\]
By the prime number theorem, $\pi(\sqrt n)=O(\sqrt n/\log n)$, hence
\[
\pi(\sqrt n)^2=O\!\left(\frac{n}{(\log n)^2}\right)
=o\!\left(\frac{n}{\log n}\right),
\qquad
\pi(\sqrt n)=o\!\left(\frac{n}{\log n}\right).
\]
Therefore
\[
\frac{\pi(\sqrt n)\bigl(\pi(\sqrt n)+1\bigr)}{2}
=o\!\left(\frac{n}{\log n}\right).
\]
Combining this with \eqref{eq:pi2-asymp} at $x=n$ yields
\[
|S|=\frac{n\log\log n}{\log n}+M\frac{n}{\log n}+o \left(\frac{n}{\log n}\right).
\qedhere\]
\end{proof}


















\end{document}

